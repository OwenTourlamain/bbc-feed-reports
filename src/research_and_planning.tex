\documentclass[12pt,titlepage]{article}
\linespread{1.5}
\usepackage{fontspec}
\usepackage{pgfgantt}
\usepackage{multicol}
\usepackage{cite}
\usepackage
[
        a4paper,% other options: a3paper, a5paper, etc
        left=4cm,
        right=2.5cm,
        top=2.5cm,
        bottom=2.5cm
]
{geometry}
\setmainfont{Arial}
\pagenumbering{arabic}

\title{BBC Feed - A Novel Interface for News Consumption Inspired by Social Media}
\date{2021}
\author{Owen Tourlamain}

\begin{document}

\begin{titlepage}
	\begin{center}
		\vspace*{1cm}

		\begingroup
      \fontsize{24}{30}\selectfont
      \textbf{BBC Feed}
    \endgroup

    \begingroup
      \fontsize{18}{22}\selectfont
      A Novel Interface for News Consumption Inspired by Social Media
    \endgroup

		\vspace{2cm}

		\textbf{Owen Tourlamain}

		\vfill

		A thesis submitted in partial fulfilment of the requirements of Birmingham City University for the degree of Master of Science

		\vspace{0.8cm}

    \begin{multicols}{2}

		\includegraphics[width=0.4\textwidth]{../img/bcu.jpg}

		Computing, Engineering and the Built Environment\\
		Birmingham City University\\
		United Kingdom\\
    \columnbreak

    \includegraphics[width=0.4\textwidth]{../img/bbcnl.jpg}

		BBC News Labs\\
		BBC\\
		United Kingdom\\

    \end{multicols}

    May 2021

	\end{center}
\end{titlepage}

\tableofcontents
\newpage

\section{Research Question}

How could social media delivery techniques be used in a digital news consumption platform to increase user engagement?

\section{Aims and Objectives}

  \subsection{Project Overview}

  This project aims to improve upon the flexibility and user engagement of traditional online news platforms, specifically those used by BBC News. The digital news landscape is changing rapidly, a 2019 report from Ofcom found that people in the UK are increasingly using 

  Modern news consumption started with the newspaper. Despite moving to online
  platforms these roots can still be seen in the user interfaces of digital news
  platforms such as the BBC News website and mobile app. These interfaces
  broadly function by providing categories of content to browse, this makes
  these interfaces good for researching topics and for getting an overview of
  recent news within a topic \cite{dummy}. Social media instead often delivers
  content to users through an infinitely scrolling feed. This style of interface
  fits well with short bursts of interaction, such as while waiting for a kettle
  to boil or waiting for a bus. In these scenarios users often want to consume
  content without having to choose a category or risk running out of content.

  This style of interface removes the need for the end user to decide what
  content they consume at the point of consumption, instead they control what
  content is presented to them by following, liking, subscribing to or otherwise
  choosing to receive content from a number of sources. From this user input a
  social media platform will choose exactly which content to provide to a user,
  and in which order. The specifics of how these decisions are made are closely
  guarded secrets, and as such will not be investigated here.

  Moving to a social media style of interface would also allow

  \subsection{Deliverables and Goals}

\section{Background and Rationale}

  \subsection{Previous Work in the Field}

  \subsection{Motivation for the Project}

\section{Literature Review}

  \subsection{Interfaces for Online News Consumption}

  \subsection{Infinite Scrolling User Interfaces}

  \subsection{Social Media Design for Increasing User Engagement}

\section{Methodology}

  \subsection{Development Methodology}

  \subsection{Project Managment}

\section{Project Timeline}

\bibliography{research_and_planning}{}
\bibliographystyle{plain}

\end{document}
