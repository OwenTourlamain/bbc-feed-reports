\documentclass[12pt,titlepage]{article}
\linespread{1.5}
\usepackage{fontspec}
\usepackage{pgfgantt}
\usepackage{multicol}
\usepackage{moreverb}
\usepackage{cite}
\usepackage
[
        a4paper,% other options: a3paper, a5paper, etc
        left=4cm,
        right=2.5cm,
        top=2.5cm,
        bottom=2.5cm
]
{geometry}
\setmainfont{Arial}
\pagenumbering{arabic}

\immediate\write18{texcount -sum -1 \jobname.tex | xargs > \jobname.wc}
\newcommand\wordcount{\input{\jobname.wc}}

\begin{document}

\begin{titlepage}
	\begin{center}
		\vspace*{1cm}

		\begingroup
      \fontsize{24}{30}\selectfont
      \textbf{BBC Feed}
    \endgroup

    \begingroup
      \fontsize{18}{22}\selectfont
      A Novel Interface for News Consumption Inspired by Social Media
    \endgroup

		\vspace{2cm}

		\textbf{Owen Tourlamain}

		\vfill

		A thesis submitted in partial fulfilment of the requirements of Birmingham City University for the degree of Master of Science

		\vspace{0.8cm}

    \begin{multicols}{2}

		\includegraphics[width=0.4\textwidth]{../img/bcu.jpg}

		Computing, Engineering and the Built Environment\\
		Birmingham City University\\
		United Kingdom\\
    \columnbreak

    \includegraphics[width=0.4\textwidth]{../img/bbcnl.jpg}

		BBC News Labs\\
		BBC\\
		United Kingdom\\

    \end{multicols}

    \today - \wordcount words

	\end{center}
\end{titlepage}

\tableofcontents
\newpage

\section{Research Question}

How could social media delivery techniques be used in a digital news consumption platform to increase user engagement?

\section{Aims and Objectives}

  \subsection{Project Overview}

  This project aims to improve upon the flexibility and user engagement of
  traditional online news platforms, specifically those used by BBC News. The
  digital news landscape is changing rapidly, a report from Ofcom found that
  people in the UK are increasingly using digital platforms to consume news,
  often through the use of social media. This project proposes that adapting
  user interface techniques from social media and applying them to a news app
  would help to keep users on the news platform. The system created will also
  aim to be flexible so that it can adapt to future changes in the digital news
  world.

  The project will be content-agnostic so that it can work with any future media
  that may be devised. To achieve this the project will be divided into three
  parts: the feed, the catalogue and the scrapers. The feed will be an app that
  users can interact with. It will need to intelligently select content from the
  catalogue and present it to the user for consumption. The catalogue will store
  content that can be displayed, this will aim to be fast to access, and well
  organised to ensure a smooth flow of content to the users. The scrapers will
  create the content for the catalogue by connecting to various BBC news data
  sources. They will read this content, translate it into a form suitable for
  the catalogue and store it. Each one will have to be custom made for the
  various content sources.

  \subsection{Key Terminology}

  \textbf{Infinite scrolling}: A user interface paradigm that presents content in a
  scrollable window. Before the end of the content is reached more content is
  loaded, resulting in the content appearing infinite to the user. This is often
  seen in social media, for example Facebook's newsfeed, and Twitter's timeline.

  \textbf{News Content}: For the purposes of this study news content will be
  defined as text, audio, photos or videos that are created or distributed by a
  news agency. These can come from a variety of sources ranging from broadcast
  news programmes to stock photos used within an article.

  \subsection{Deliverables and Goals}

  The deliverables for this project will be the three components highlighted
  above, as well as all documentation require to understand, maintain, deploy
  and use the system. The documentation will be provided as HTML pages alongside
  documentation within the codebase. All code produced will be written in
  Python.

    \subsubsection{The Feed}

    This deliverable will be a Python Flask application that can either be run
    locally or deployed to a web server. It will have accompanying scripts that
    install any pre-requisites for running the application. This deliverable
    will depend on having a valid catalogue to read from. If the hosted
    catalogue is unavailable there will be options to read from a local version.

    \subsubsection{The Catalogue}

    This deliverable will take the form of an SQL database schema. The database
    that it describes will be hosted on an AWS RDS instance. As this may not be
    available at all times the feed will have the ability to deploy a local
    database to read from for testing purposes.

    \subsubsection{The Scrapers}

    This part of the project consists of several independent scripts, each will
    be tailored to a specific BBC data source. The specifics for language,
    running and hosting will vary between the scripts. They will each come with
    appropriate documentation.

  \subsection{Out of Scope Issues}

  This project is primarily focused on the engineering challenges of developing
  this system, as such it will not be attempting to measure user experiences in
  any way. The project will also not explore algorithms for choosing which
  content to display to users, it will however provide capacity for this
  functionality to be added at a later date. These topics are highly important
  to the problem space being explored, however they each could constitute a
  project in their own right and would spread the focus of this project too
  thin. These two issues would make for excellent follow up projects however so
  some comments may be made where relevant to future research.

\section{Background and Rationale}

  \subsection{Overview of the Problem Space}

  Modern news consumption started with the newspaper. Despite moving to online
  platforms these roots can still be seen in the user interfaces of digital news
  platforms such as the BBC News website and mobile app. These interfaces
  broadly function by providing categories of content to browse, this makes
  these interfaces good for researching topics and for getting an overview of
  recent news within a topic \cite{dummy}. Social media instead often delivers
  content to users through an infinitely scrolling feed. This style of interface
  fits well with short bursts of interaction, such as while waiting for a kettle
  to boil or waiting for a bus. In these scenarios users often want to consume
  content without having to choose a category or risk running out of content.

  This style of interface removes the need for the end user to decide what
  content they consume at the point of consumption, instead they control what
  content is presented to them by following, liking, subscribing to or otherwise
  choosing to receive content from a number of sources. From this user input a
  social media platform will choose exactly which content to provide to a user,
  and in which order. The specifics of how these decisions are made are closely
  guarded secrets, and as such will not be investigated here.

  While social media allows users to access news content in this way, it is
  often preferential to deliver news to users on a first party platform, such as
  the app or website of the news agency. This gives the producers of the content
  more control over how it is presented and consumed, as well as what other
  content may be presented to the user alongside the current content. This
  project aims to create an interface inspired by social media, but completely
  controlled by the news agency.

  This also gives the creators of this content more flexibility to experiment
  with new content types, they are not limited to the narrow selection of media
  that the various social platforms offer. This feed can also offer features
  such as A/B testing that would enable the content producers to test the
  effectiveness of new media types.

  When creating printed media, such as a newspaper, each article or photograph
  that is included takes up space that could have been used for something else.
  Because of this it is key to only select content that will have the greatest
  impact, and testing new content is risky. In an infinite scrolling style of
  interface however there is no limit on the amount of content that can be shown
  to a user, as such inserting new media forms to test them becomes more viable
  as if the user is not interested they can scroll past and move on to the next
  piece of content.

  \subsection{Motivation for the Project}

  This last issue is the primary inspiration for this project. As part of the
  BBC News Labs team I worked on the SlicerAV and Live Segment Notifications
  (LSN) projects. SlicerAV takes broadcast news programmes and
  automatically breaks them up into "slices" of short form media. Once this
  project was functional the next question was how to deliver this content to
  end users. For testing purposes we decided to tweet the slices as we felt that
  they fit best on to a social media platform, and twitter worked best for our
  use-case.

  The LSN project aimed to inspect news broadcasts just before they go live and
  notify users about upcoming content that may be interesting to them. This
  worked well, however we didn't have a good platform for hosting this content
  internally. This raised to possibility of a feed that would hold all the
  slices that a user had been notified about for them to browse. We then took
  that idea and realised it might fit well if applied to a news homepage, where
  sliced content could be surfaced alongside regular articles and videos.

  Newslabs is focused on innovating news production and consumption, as such
  they have several projects that could benefit from a flexible user interface
  to surface their content. One such project is Graphical Story Telling (GST)
  which aims to take articles and turn them into a series of graphics with
  overlaid text that can be swiped through. This content is inspired by social
  media and would fit perfectly into this project. This provided the idea to
  make a flexible feed that can present a wide variety of content in one place,
  with functionality for personalisation and testing.

\section{Literature Review}

  \subsection{Interfaces for Online News Consumption}

  \subsection{Infinite Scrolling User Interfaces}

  \subsection{Social Media Design for Increasing User Engagement}

\section{Methodology}

  \subsection{Development Methodology}

  \subsection{Project Management}

\section{Project Timeline}


\bibliography{research_and_planning}{}
\bibliographystyle{plain}

\end{document}
